\section{Tuesday, May 7}

\todaybox{ Complex numbers, and complex arithmetic.}


\begin{defbo}{The Complex Numbers}{complexNumbers}
The imaginary unit $i$ is a number such that $i^2 = -1$. A complex number is a number of the form $a + bi$, where $a,b\in \R$. The set of complex numbers is $\C = \{a + bi| a,b\in \R\}$.
\end{defbo}

\begin{note}The name "imaginary" is a misnomer. When mathematicians first started thinking about complex numbers, $i$ was treated at best as a calculation trick and at worst as pure nonsense. The name was originally chosen as an insult.

Now, we recognize that the concept is real in the same way any other advanced mathematics is. These actually exist, and we can formally design a system that exhibits this behavior.

Beyond that, however, complex numbers are actually useful in real life. For example, the mathematics behind electomagnetism is based on working with complex numbers.\end{note}

\begin{ex}{}{} For example, $2 + 3i$, $\pi - ie^2$, and $1$ are all complex numbers.

Why would $1$ be a complex number? Isn't it real? When we write $1$ in this context, we mean $1 + 0i$. In this way, we can think of every real number $r$ as a complex number as well: $r = r + 0i$.
\end{ex}

\subsection{Complex Algebra}

Let's talk about how to manipulate complex numbers. Our overarching goal is to develop some notion of calculus. This requires us to be able to do algebra on $\C$.

\begin{defbo}{Real and Imaginary Parts}{ReAndIm}\index{Algebra!real part}\index{Algebra!imaginary part} 
Let $a+ bi\in \C$. Then the {\bf real and imaginary parts} of $a+bi$ are:

$$\RE(a+bi) = a$$
$$\IM(a+bi) = b$$
\end{defbo}

\begin{ex}{}{} Consider $z = 3 + 4i$. The real part of $z$ is $3$, and the imaginary part is $4$. Notice: $4$, not $4i$. The imaginary part of $z$ is still a real number.
\end{ex}


\begin{defbo}{Addition}{addition}\index{Algebra!addition}
 Let $a+bi, c+di \in \C$. Then:

$$(a+bi) + (c+di) = (a+c) + (b+d)i$$
\end{defbo}

So adding $z,w \in \C$ is done by adding together the real parts of $z,w$, and adding together the imaginary parts.

\begin{defbo}{Multiplication}{multiplication}\index{Algebra!multiplication} 
Let $a + bi, c+di \in \C$. Then:

$$(a+bi)(c+di) = ac - bd + (ad + bc)i$$
\end{defbo}

Why would we choose this definition? Well, we want complex multiplication to satisfy the "distributivity property": $(a+b)c = ac+bc$ and $a(b+c) = ab + ac$. If we require these to hold, then we are forced to conclude that:

\begin{align*}(a+bi)(c+di) &= (a+bi)c + (a+bi)di\\
&= (ac + bic) + (adi + bidi)\\
&= ac + bci + adi + bdi^2\\
&= ac + bci + adi - bd\\
&= (ac -bd) + (ad + bc)i
\end{align*}

Complex multiplication and addition satisfy a whole bunch of properties, specifically what are called the field axioms.


\begin{thmbo}{The Field Axioms}{fieldAxioms}
The complex numbers satisfy the following properties. For all $u, w, z \in \C$:

\begin{enumerate}
\item $w + z = z + w$
\item $u + (w + z) = (u+ w) + z$
\item $z + 0 = z$
\item If $z = x + iy$, then $-z = (-x) + i(-y)$ satisfies that $z + (-z) = 0$.
\item $wz = zw$
\item $u(wz) = (uw)z$
\item $1z = z$
\item For any $z\in \C$ with $z\ne 0 + 0i$, there exists some $w\in \C$ with $zw = 1$. 
\item $u(w+z) = uw + uz$ and $(u+w)z= uz + wz$
\end{enumerate}
\end{thmbo}

\begin{proof} Proof is left as an exercise.\end{proof}

\begin{ex}{}{} Let $z = 2 + 7i$, $w = 4 - 3i$. Find $w^2 - zw$.

Well, to make life simpler, we can factor (using distributivity):
$$w^2 - zw = w(w-z) = (4-3i)[(4-3i) - (2 + 7i)] = (4-3i)(2 - 10i) = (8 - 30) + (-40 - 6)i = -22 - 46i$$

\end{ex}

What about division? First, what is division? What does $\frac{1}{z}$ mean? 

\begin{defbo}{Multiplicative Inverse}{multinv}
Let $z\in \C$. We say that $w = \frac{1}{z}$ if $zw =1$. In this situation, $w$ is a {\bf multiplicative inverse} for $z$.
\end{defbo}

So how do we find a multiplicative inverse for $z$? To do that, we're going to need to introduce two new ideas, the complex conjugate and the modulus:

\begin{defbo}{Complex Conjugate}{conjugate}\index{Algebra!complex conjugate} 
Let $a+bi \in \C$. Then the {\bf complex conjugate} of $a+bi$ is:

$$\overline{a+bi} = a-bi$$
\end{defbo}

\begin{defbo}{Modulus}{modulus}\index{Algebra!modulus} 
Let $a+bi \in \C$. Then the {\bf modulus} of $a+bi$ is the real number:

$$|a+bi| = \sqrt{a^2 + b^2}$$
\end{defbo}

How does this help us define division? It turns out, these are exactly the building blocks we need:

\begin{lem}Let $a+bi\in \C$ be non-zero (i.e., at least one of $a,b$ is not 0). Then $\frac{1}{z} = \frac{a}{a^2 + b^2} - \frac{b}{a^2 + b^2}i$. Another way of writing this is that $\frac{1}{z} = \frac{\overline{z}}{\left|z\right|^2}$, where we understand that division by a real number means division on both the real part and imaginary part.\end{lem}

