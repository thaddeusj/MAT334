\section{Thursday, July 11}

\todaybox{Having discussed how to integrate around poles, we need to handle our other two kinds of disconinuities. Both of these will require us to discuss power series.

We begin today with a brief refresher on sequences and series.}

\begin{defbo}{Sequence}{}\index{Sequence} A sequence in $\C$ is an infinite list of complex numbers $a_k,a_{k+1},\dots$. We will write this as $(a_n)_{n = k}^\infty$.

We say that the sequence $(a_n)_{n = k}^\infty$ converges to $L$, or that $\lim_{n\rightarrow \infty} a_n = L$, or even that $a_n \rightarrow L$, if:

$$\forall \varepsilon > 0, \exists N\in \N \text{ such that } n > N \implies |a_n - L|< \varepsilon$$

If a sequence does not converge to any limit $L$, then it diverges.
\end{defbo}

We aren't going to spend a lot of time discussing sequences. We only care about them in the context of series.

\begin{defbo}{Series}{}\index{Infinite Series} Let $(a_n)_{n = k}^\infty$. Define a new sequence $(S_n)_{n = k}^\infty$ by:

$$S_n  = \sum_{j = k}^n a_j$$

This is called the $n$th partial sum of the sequence $(a_n)_{n = k}^\infty$. The infinite series $\sum_{n = k}^\infty a_n$ is defined to be $\sum_{n = k}^\infty a_n = \lim_{n \rightarrow \infty} S_n$, if the sequence converges. In that case, we say the sequence converges as well.
\end{defbo}

Let's look at a couple of examples of convergence.

\begin{ex}{}{}\index{Series!geometric}\index{Geometric series} The {\bf geometric series} $\sum_{n = 0}^\infty z^n$ converges to $\frac{1}{1-z}$ if $|z| < 1$ and diverges if $|z| > 1$.

Consider the $n$th partial sum:

$$S_n = 1 + z + z^2 + \dots + z^n$$

This is a finite geometric series, which is equal to $\frac{1 - z^{n+1}}{1-z}$. To see this, notice that:

\begin{align*}(1-z)(1+ z + \dots + z^n) &= (1-z) + z(1-z) + z^2(1-z) + \dots + z^n(1-z) \\
&= 1 - z + z - z^2 + z^2 - \dots - z^n + z^n - z^{n+1}\\
&= 1-z^{n+1}
\end{align*}

So for $z\ne 1$, $S_n = \frac{1-z^{n+1}}{1-z}$. We now take the limit of this sequence:

$$\lim_{n\rightarrow \infty} \frac{1 - z^{n+1}}{1-z} = \frac{1 - \lim_{n \rightarrow \infty} z^{n+1}}{1-z}$$

If $|z| < 1$, notice that $|z^{n+1}|\rightarrow 0$ (while we could show this formally, this is something you saw in first year calc.) Furthermore, if $|z| > 1$, then $|z^{n+1}|\rightarrow \infty$. This tells us that if $|z| < 1$ that $z^{n+1}\rightarrow 0$ and if $|z| > 1$ that $z^{n+1}\rightarrow \infty$. As such:

$$\lim_{n\rightarrow \infty} S_n = \begin{cases} \frac{1}{1-z}, & |z| < 1\\\infty, &|z| > 1\end{cases}$$
\end{ex}

This example is going to be very useful. Make sure you know this formula.

For a lot of series, actually calculating the limit can be quite hard. For example, how would you show that $\sum_{n = 0}^\infty \frac{2^n}{n!} = e^2$, or that $\sum_{k = 0}^\infty \frac{(-1)^k}{(2k+1)!} = \sin(1)$?

For the moment, let's instead focusing on determining if a series converges at all. We begin by recalling a couple of convergence results that you should have already seen for $\R$.

\begin{thmbo}{The Divergence Test}{}\index{Series!divergence test} If $\lim_{n\rightarrow \infty} a_n \ne 0$, then $\sum_{n = k}^\infty a_n$ diverges.

The converse is false.
\end{thmbo}
\begin{proof} We prove the contrapositive instead. Suppose $\sum_{n = k}^\infty a_n$ converges.

Notice that $a_n = S_n - S_{n-1}$. As such, $\lim_{n\rightarrow \infty} a_n = \lim_{n\rightarrow \infty} S_n - \lim_{n\rightarrow \infty} S_{n-1}$. Notice that the sequences $(S_n)_{n = k}^\infty$ and $(S_{n-1})_{n = k+1}^\infty$ have the same limit. Indeed, these are actually the same sequence, just indexed differently! So: 

$$\lim_{n\rightarrow \infty} a_n = \sum_{n = k}^\infty a_n - \sum_{n = k}^\infty a_n = 0$$

For the converse, we claim that $\sum_{n = 1}^\infty \frac{1}{n}$ diverges. The idea is to write the sum as:

$$1 + \frac{1}{2} + \underbrace{\frac{1}{3} + \frac{1}{4}}_{< \frac{1}{4} + \frac{1}{4} = \frac{1}{2}} + \underbrace{ \frac{1}{5} + \frac{1}{6} + \frac{1}{7} + \frac{1}{8}}_{< \frac{4}{8} = \frac{1}{2}} + \dots$$

More generally, notice that $\sum_{n = 2^k + 1}^{2^{k+1}} \frac{1}{n} > \sum_{n = 2^k + 1}{2^{k+1}} \frac{1}{2^{k+1}} = \frac{1}{2}$. As such:

$$S_{2^{m+1}} = \sum_{n = 1}^{2^{m+1}} \frac{1}{n} =1 + \frac{1}{2} +  \sum_{k = 1}^m \sum_{n = 2^k + 1}^{2^{k+1}} \frac{1}{n} \ge 1 + \frac{1}{2} + \frac{m}{2}$$

Therefore, $\lim_{n\rightarrow \infty} S_n = \lim_{m\rightarrow \infty} 1 + \frac{m+1}{2} = \infty$. So the series diverges.

This is called the harmonic series, and is the classic example of a series whose terms go to $0$ be which diverges anyway.
\end{proof}

This is not the only example of a series whose terms go to $0$ but which diverges. There are plenty of interesting examples of this behavior. For example, $\displaystyle\sum_{p \text{ prime}} \frac{1}{p}$ also diverges, although much more slowly. While outside the scope of this course, this is traditionally shown using the Prime Number Theorem, which is equivalent to saying that the $n$th prime $p_n$ is approximately $n\ln(n)$ for $n$ very large.

Another way to tell if a series is convergent is if it is absolutely convergent.

\begin{defbo}{Absolutely Convergent Series}{}\index{Series!absolutely convergent} A series $\sum_{n = k}^\infty a_n$ is called absolutely convergent if $\sum_{n = k}^\infty |a_n|$ converges.\end{defbo}

\begin{thmbo}{}{}Absolutely convergent series converge.\end{thmbo}

\begin{proof} See \ref{thm:absConv} in the appendices for a proof. It is a technical proof, using the notion of Cauchy sequences. I have included it for completeness's sake.\end{proof}

With this new notion, we can prove a very powerful convergence test.

\begin{thmbo}{D'Alembert's Ratio Test}{}\index{Series!ratio test} Suppose $\displaystyle\lim_{n\rightarrow \infty} \left| \frac{a_{n+1}}{a_n}\right| = L$.
\begin{itemize}
\item If $L < 1$, then $\sum_{n = k}^\infty a_n$ converges absolutely. (And therefore converges.)

\item If $L > 1$, the series diverges.
\end{itemize}
\end{thmbo}
 
\begin{proof} Suppose $\displaystyle\lim_{n\rightarrow \infty} \left| \frac{a_{n+1}}{a_n}\right| = L$.

Suppose $L < 1$. Choose $\varepsilon > 0$ so that $L < L + \varepsilon < 1$. Since $\displaystyle\lim_{n\rightarrow \infty} \left| \frac{a_{n+1}}{a_n}\right| = L$, there exists $N \in \N$ with $N > k$ such that if $n \ge N$, then $\displaystyle \left| \frac{a_{n+1}}{a_n}\right| < L + \varepsilon$.

As such, $|a_{N+j}| < (L + \varepsilon)|a_{N + j - 1}|$ for all $j \ge 1$. We can use this iteratively to conclude that $|a_{N+j}| < (L + \varepsilon)^j|a_N|$.

We see then that $\sum_{n = k}^\infty |a_n| < \sum_{n = k}^{N-1} |a_n| + \sum_{j = N}^\infty (L + \varepsilon)^j|a_N|$. Since $L + \varepsilon < 1$, this geometric series converges.

Now, the Comparison Theorem (from first year calculus) tells us that if $0 \le a_n \le b_n$ and $\sum_{n = k}^\infty b_n$ converges, then so does $\sum_{n = k}^\infty a_n$. So, by the Comparison Theorem, $\sum_{n = k}^\infty |a_n|$ converges.

Therefore, $\sum_{n=k}^\infty a_n$ converges absolutely.

Now, supposing that $L > 1$, we choose $\varepsilon$ so that $1 < L - \varepsilon < L$. Since $\displaystyle\lim_{n\rightarrow \infty} \left| \frac{a_{n+1}}{a_n}\right| = L$, there exists $N \in \N$ with $N > k$ such that if $n \ge N$, then $\displaystyle \left| \frac{a_{n+1}}{a_n}\right| > L - \varepsilon$.

In this case, we know that for $j \ge 0$, $|a_{N + j}| > (L - \varepsilon)^j|a_N|$. However, as $j\rightarrow \infty$,  $(L_\varepsilon)^j \rightarrow \infty$. As such, $\lim_{n \rightarrow \infty} a_n = \infty$. By the Divergence Test, $\sum_{n = k}^\infty a_n$ diverges.

\end{proof}

This is going to be an extremely useful test for us. In fact, when dealing with power series (as we will be very shortly), this test is the most useful.

\begin{ex}{}{} The series $\sum_{n = 0}^\infty \frac{z^n}{n!}$ converges for all $n$.

To see this, we apply the ratio test. We need to be careful here. Remember that in the ratio test, $a_n$ represents the $n$th term of the series. It does not represent the coefficient of $z^n$.

So, in this case, $a_n = \frac{z^n}{n!}$. And therefore:

$$\lim_{n\rightarrow \infty} \left| \frac{\frac{z^{n+1}}{(n+1)!}}{\frac{z^n}{n!}}\right| = \lim_{n\rightarrow \infty} \frac{|z|}{n+1} = 0$$

Since $L = 0 < 1$ for all $z\in \C$, we see that the series converges absolutely for each $z\in \C$.
\end{ex}

\begin{ex}{}{} Find an $R\ge 0$ such that if $|z| < R$, then $\sum_{n = 0}^\infty \frac{z^n}{2^n}$ converges, and if $|z| > R$, then the sum diverges.

This turns out to follow right away from the ratio test. We apply:

$$\lim_{n\rightarrow \infty} \left| \frac{\frac{z^{n+1}}{2^{n+1}}}{\frac{z^n}{2^n}}\right| = \lim_{n\rightarrow \infty} \frac{|z|}{2}$$

Now, if $\frac{|z|}{2} < 1$, the ratio test tells us that the series converges. And if $\frac{|z|}{2}> 1$, the series diverges. In other words, $R = 2$ works.
\end{ex}

\subsection{Power Series}

Now that we've discussed series and developed the relevent tools, we can now talk about power series.

\begin{defbo}{Power series}{}\index{Power series}\index{Series!power series}A power series centered at $z_0$ is a function $f(z)$ given by:

$$f(z) = \sum_{n = 0}^\infty a_n(z-z_0)^n$$

The domain of this function is every point where this series converges, and contains at least $z_0$.

For some function $g(z)$, we say that $g(z)$ is represented by the power series $\sum_{n = 0}^\infty a_n(z-z_0)^n$ on a domain $D$ if $g(z) = \sum_{n = 0}^\infty a_n(z-z_0)^n$ on $D$.
\end{defbo}

We've already seen one example of a power series.

\begin{ex}{}{} $\frac{1}{1-z}$ is represented by the power series $\sum_{n = 0}^\infty z^n$ on $B_1(0)$.\end{ex}

One of the most crucial facts moving forward is that holomorphic functions can be represented by power series. This is a drastic difference between the complex and real theory. There are infinitely differentiable functions over $\R$ which cannot be meaningfully represented by power series. That is not the case over $\C$.

\begin{thmbo}{}{holoAna} Suppose $f(z)$ is holomorphic on a domain $D$, and $z_0 \in D$. Then there exists some $R > 0$ such that if $|z- z_0| < R$, then:

$$f(z) = \sum_{n = 0}^\infty \frac{f^{(n)}(z_0)}{n!}(z-z_0)^n$$
\end{thmbo}

\begin{proof} Suppose $f(z)$ is holomorphic on $D$. Let $R > 0$ such that $B_R(z_0) \subset D$.

Let $w \in B_R(z_0)$. Our goal is to show that $f(w) = \sum_{n = 0}^\infty \frac{f^{(n)}(z_0)}{n!}(w-z_0)^n$.

Choose $r \in \R$ such that $|w - z_0| < r < R$. This can be done since $w \in B_R(z_0) \implies |w - z_0| < R$. Let $\gamma_r$ be the circle of radius $r$ centered at $z_0$, travelled once counterclockwise. By the Cauchy Integral Formula:

$$f(w) = \frac{1}{2\pi i } \int_{\gamma_r} \frac{f(z)}{z-w}dz$$

For the moment, let us consider $\frac{1}{z-w}$. We can rewrite this as:

$$\frac{1}{z-w} = \frac{1}{(z-z_0) - (w - z_0)} = \frac{1}{(z-z_0)}\frac{1}{1- \frac{w - z_0}{z-z_0}}$$

Now, since $|w - z_0| < r$ and $|z - z_0| = r$ (since $z$ is on $\gamma_r$), we know that $\left|\frac{w-z_0}{z-z_0}\right| < 1$. So, our geometric series formula gives that:

$$\frac{1}{z-w} = \frac{1}{z-z_0} \sum_{n = 0}^\infty \left(\frac{w - z_0}{z-z_0}\right)^n = \sum_{n = 0}^\infty \frac{(w-z_0)^n}{(z-z_0)^{n+1}}$$

And therefore, $f(w) = \frac{1}{2\pi i} \int_{\gamma_r} \sum_{n = 0}^\infty (w-z_0)^n\frac{f(z)}{(z-z_0)^{n+1}}dz$. By uniform convergence of the series (see theorem \ref{thm:1/1-z} in the appendices for the very technical proof), we have that:

\begin{align*}f(w) &= \frac{1}{2\pi i} \int_{\gamma_r} \sum_{n = 0}^\infty (w-z_0)^n\frac{f(z)}{(z-z_0)^{n+1}}dz\\
&= \sum_{n = 0}^\infty \frac{1}{2\pi i}\int_{\gamma_r} (w-z_0)^n\frac{f(z)}{(z-z_0)^{n+1}} dz\\
&=\sum_{n = 0}^\infty \frac{(w-z_0)^n}{2\pi i}\int_{\gamma_r} \frac{f(z)}{(z-z_0)^{n+1}} dz\\
&\hspace{-4pt}\stackrel{CIF}{=} \sum_{n = 0}^\infty \frac{(w-z_0)^n}{2\pi i}\frac{2\pi i f^{(n)}(z_0)}{n!}\\
&= \sum_{n = 0}^\infty \frac{f^{(n)}(z_0)}{n!}(w-z_0)^n
\end{align*}

\noin as desired. Since $w$ was chosen arbitrarily in $B_R(z_0)$, the power series expansion holds on $B_R(z_0)$.

\end{proof}

We finish today by calculating a couple of examples of power series.

\begin{ex}{}{} Consider $f(z) = e^z$. Find the power series expansion for $e^z$ centered at $z_0$. For which $z\in \C$ is it valid?

Well, $f^{(n)}(z) = e^z$, and so the power series expansion of $e^z$ at $z_0$ is:

$$\sum_{n = 0}^\infty \frac{e^{z_0}}{n!}(z-z_0)^n$$

Now, the theorem above was valid on $B_R(z_0)$ where $B_R(z_0)$ is contained in a domain where $f(z)$ is holomorphic. However, $e^z$ is entire, and so all $R$ satisfy this condition. As such, the power series expansion is valid on $B_R(z_0)$ for any $R$. Since for each $w\in \C$, there exists $R$ so that $w\in B_R(z_0)$, it follows that this power series expansion is valid on $\C$.
\end{ex}