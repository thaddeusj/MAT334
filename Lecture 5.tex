\section{Tuesday, May 21}

\todaybox{We will discuss the complex trigonometric functions. More importantly, we're going to begin our discussion of complex logarithms.}

Last class, we defined the complex exponential function:

$$e^{x+iy} = e^xe^{iy} = e^x(\cos(y) + i\sin(y))$$

Let's talk about this some more.

\begin{ex}{}{} Let $w = 1 + 4i$. Solve the equation $e^z = w$.

Let $z = x + iy$. Then $e^xe^{iy} = 1+ 4i$. This tells us that:
$$e^x = |w| = \sqrt{17}$$

So we conclude that $x = \ln(\sqrt{17})$.

Further, the expression $w = e^xe^{iy}$ tells us that $y$ is an argument for $w$! So, we need to find the arguments for $w$. We see that $w = \sqrt{17}e^{i\arctan(4)}$.

Therefore, $z = \ln(\sqrt{17}) + i\arctan(4)$.

\exercisebox{There is an error in this argument. What is it?}

 We know that $y$ is AN argument for $w$. Not that $y$ is this particular argument for $w$. Instead, we can only conclude that $y = \arctan(4) + 2k\pi$ for some $k\in \Z$, and therefore $z = \ln(\sqrt{17}) + i(\arctan(4) + 2k\pi)$.
\end{ex}

\begin{note} $\ln(\sqrt{17}) + i\arctan(4)$ and $\ln(\sqrt{17}) + i(\arctan(4) + 2\pi)$ are different complex numbers! While $\arctan(4)$ and $\arctan(4) + 2\pi$ point in the same direction as angles, $y$ is not an angle. $y$ is the vertical component of the complex number $z$.

This distinction is important. $y$ is not an angle, and so we can't ignore this $2k\pi$.\end{note}

\begin{ex}{}{hardexpo} Solve the equation $e^{iz} - e^{-iz} = 2i$.

It is possible to solve this equation by setting $z = x+ iy$, expanding into rectangular form, and then solving the resulting equations. However, this turns out to be fairly difficult.

Instead, let $e^{iz} = w$. Then we have:
$$w - \frac{1}{w} = 2i$$

After some quick algebra, we can rearrange this to become:
$$w^2 - 2iw - 1 = 0$$

Now, from your homework, we know that we can solve this using the quadratic formula:
$$w = \frac{2i \pm (-4 + 4)^{\frac{1}{2}}}{2} = i$$

So, the solutions $z$ to the original equation satisfy $e^{iz} = i$. Let $z = x+iy$. Then $e^{-y + ix} = i$. This gives $e^{-y} = 1$, so $y = 0$.

Also, we know that $x$ is an argument for $i$, so $x = \frac{\pi}{2} + 2k\pi$ for $k\in \Z$. Therefore, $e^{iz} - e^{-iz} = 2i$ if and only if $z = \frac{\pi}{2} + 2k\pi$ for some $k\in \Z$.

\end{ex}


\begin{ex}{}{} True or false: For any $z\in \C$, $e^z > 0$.

False. Remember, it is nonsense to say that $a > b$ if $a,b$ are complex numbers. $e^z$ can be any complex number (except for $0$), so this is a nonsense statement.

More concretely, $e^{i\pi} = -1 \not> 0$.

\end{ex}

\subsection{Complex Trigonometric Functions}

Looking at Euler's formula, $e^{i\theta} = \cos(\theta) + i\sin(\theta)$, it seems like there's a connection between trigonometric functions and the complex exponential.

If we play around with this fact, we can see:
\begin{align*}
e^{i\theta} &= \cos(\theta) + i\sin(\theta)\\
e^{-i\theta} &= \cos(-\theta) + i\sin(-\theta) = \cos(\theta) - i\sin(\theta)
\end{align*}

Adding these expressions together, we see that:
$$\cos(\theta) = \frac{e^{i\theta} + e^{-i\theta}}{2}$$
$$\sin(\theta) = \frac{e^{i\theta} - e^{-i\theta}}{2i}$$

Since this is our only connection between trigonometric functions and complex numbers, it seems reasonable to use this to define complex versions of $\cos(z)$ and $\sin(z)$.

\begin{defbo}{Trigonometric Functions}{trigFunc}\index{Function!trigonometric} The complex differentiable functions $\cos(z)$ and $\sin(z)$ are defined by:
$$\cos(z) = \frac{e^{iz} + e^{-iz}}{2}$$
$$\sin(z) = \frac{e^{iz} - e^{-iz}}{2i}$$

The other trigonometric functions $\tan(z)$, $\sec(z)$, $\csc(z)$, and $\cot(z)$ are defined exactly how you would expect.
\end{defbo}

Just like for our complex exponential, notice that if $z = x\in \R$, then $\cos(z)$ is exactly the real cosine function $\cos(x)$, and similarly for $\sin(z)$. So this isn't totally unreasonable.


Many of the usual properties of the real trigonometric functions are still satisfied by these complex functions as well. For example:

\begin{ex}{}{} We know that the real function $\cos(x)$ is $2\pi$ periodic. I.e., $\cos(x) = \cos(x + 2\pi)$. This is still true over $\C$.

To see this, note that:
$$\cos(z + 2\pi) = \frac{e^{i(z + 2\pi)} + e^{-i(z + 2\pi)}}{2} = \frac{e^{iz}e^{i2\pi} + e^{-iz}e^{-i2\pi}}{2} = \frac{e^{iz} + e^{-iz}}{2} = \cos(z)$$
\end{ex}

However, these complex functions can have some wildly different behavior as well.

\begin{ex}{}{} True or false: $-1 \le |\sin(z)| \le 1$ for any $z\in \C$.

As we saw in class, when looking at $\sin(iy)$, we see that for $y$ very large, $|\sin(iy)|$ is very large. In fact, $|\sin(iy)| \approx \frac{e^{|y|}}{2}$.

In fact, the range of $\sin(z)$ is actually $\C$!

\end{ex}

\begin{ex}{}{} Find all solutions to $\sin(z) = 1$.

We are looking for all $z$ for which $\sin(z) = \frac{e^{iz} - e^{-iz}}{2i} = 1$. Notice that this is equivalent to solving $e^{iz} - e^{-iz} = 1$. As we saw in example \ref{exa:hardexpo}, the solutions to this are precisely $z = \frac{\pi}{2} + 2k\pi$ for $k\in \Z$. So the only solutions to $\sin(z) = 1$ are real solutions.

\end{ex}





Do we also have a corresponding notion of a complex logarithm?

\subsection{The Complex Logarithm}

To begin, what is a logarithm? What does it mean to say that $w$ is a logarithm for $z$?

\begin{defbo}{Logarithms}{logarithm}\index{Logarithm} 
Let $z\in \C$. We say that $w$ is a logarithm for $z$ if $e^w = z$.
\end{defbo}

We've already seen an example of finding a logarithm. Last class, we showed that the solutions to $e^{z} = 1 + 4i$ are of them form $z = \ln(\sqrt{17}) + i(\arctan(4) + 2k\pi)$ for $k\in \Z$. So, $1 + 4i$ has many logarithms. These are precisely the $z$ listed above.


Is there a general formula for finding the logarithms of $z\in \C$, or do we need to do it by hand each time we wish to solve $e^z = w$?

\begin{thmbo}{Calculating Logarithms}{logCalc}
 Let $z = re^{i\theta}$ with $z\ne 0$. Then the logarithms of $z$ are the complex numbers $\ln(r) + i(\theta + 2k\pi)$, where $k\in \Z$. Put another way, $\log(z) = \ln|z| + i\arg(z)$, remembering that we mean this as multi-valued functions.


If $z = 0$, then $z$ has no logarithms.
\end{thmbo}

\begin{proof} First, we handle the situation where $z\ne 0$. Suppose $w = a +bi$ and $e^w = z$. Then:
$$e^ae^{ib} = re^{i\theta}$$

Taking the modulus of both sides, we see that $e^a = r$, and so $a = \ln(r)$, which is defined since $r\ne 0$.

Now, we can see that $re^{ib} = z$, and so $b$ is an argument for $z$. As we have shown before, this means that $b = \theta + 2k\pi$ for some $k\in Z$.

As for $z = 0$, notice that $|e^w| = e^a \ne 0$. However, $|z| = 0$. So, we cannot have $e^w = z$.

\end{proof}
