\section{Tuesday, July 30}

\todaybox{We begin our discussion of Laurent series, which we will use to integrate around essential singularities.}

We've discussed integrating around removable singularities and around poles, and more or less fully discussed those. But what about essential singularities? To discuss those, we need to talk about Laurent series.

\begin{defbo}{Laurent Series}{}\index{Laurent Series}\index{Series!Laurent series}
A Laurent series centered at $z_0$ is a function $f(z)$ of the form:

$$f(z) = \sum_{n = -\infty}^\infty a_n(z-z_0)^n = \sum_{n = 0}^\infty a_n(z-z_0)^n + \sum_{n = 1}^\infty \frac{a_{-n}}{(z-z_0)^n}$$

The function is defined wherever both of these series exist, and is undefined whenever one diverges.
\end{defbo}

Before we get into how this idea is useful, let's start with an example:

\begin{ex}{}{} Suppose $f(z) = \frac{1}{1-z}$. We know that on $|z| < 1$ that $f(z) = \sum_{n = 0}^\infty z^n$.

What about for $|z| > 1$? Well, we can rewrite:

$$f(z) = \frac{1}{z}\frac{1}{\frac{1}{z} - 1} = -\frac{1}{z}\frac{1}{1-\frac{1}{z} }$$

Well, we know that $|z| > 1$, so $\left|\frac{1}{z}\right| < 1$. So, we can use the power series valid on $|z| < 1$ to get that:

$$\frac{1}{1- \frac{1}{z}} = \sum_{n = 0}^\infty \left(\frac{1}{z}\right)^n$$

All together, this gives us that:

$$\frac{1}{1-z} = -\sum_{n = 1}^\infty \frac{1}{z^n}$$

\end{ex}

Alright, so let's tackle some theoretical concerns. In particular, we're interested in a few questions:

\begin{itemize}
\item Is there a radius of convergence type condition that let's us see when Laurent series converge?
\item Are Laurent series analytic?
\item Do analytic functions admit Laurent series?
\end{itemize}

Regarding the first question: we broke $\sum_{n = -\infty}^\infty a_n(z-z_0)^n$ into two sums. The sum with the positive powers:

$$\sum_{n = 0}^\infty a_n(z-z_0)^n$$

\noin is a power series. As such, it has a radius of convergence $R_1$ so that it converges on $B_{R_1}(z_0)$, and diverges when $|z-z_0| > R_1$.

To investigate the negative terms, we'll make a change of variables. Let $w = \frac{1}{z-z_0}$. Then we have:

$$\sum_{n = 1}^\infty \frac{a_{-n}}{(z-z_0)^n} = \sum_{n = 1}^\infty a_{-n}w^n$$

\noin which is a power series in $w$. As such, it has a radius of convergence $R_2$ so that it converges when $|w| < R_2$ and diverges when $|w| > R_2$. As such, this series converges when $|z-z_0| > \frac{1}{R_2}$ and diverges when $|z-z_0| < \frac{1}{R_2}$.

As such, there are two radii $r_1$ and $r_2$ such that the series converges when $r_1 < |z-z_0| < r_2$. Be careful: this does not automatically imply that $r_1 < r_2$. We need $r_1 < r_2$ for the series to converge anywhere, but we can easily find series such that $r_1 > r_2$, so that when the positive powers converge, the negative powers diverge.

Regarding analyticity:

\begin{thmbo}{}{} If $f(z) = \sum_{n = -\infty}^\infty a_n(z-z_0)^n$ converges when $R_1 < |z-z_0| < R_2$ and $R_1 < R_2$, then:

$$f'(z) = \sum_{n = -\infty}^\infty na_n(z-z_0)^{n-1}$$

\noin In particular, this new series also converges when $R_1 < |z-z_0| <R_2$.
\end{thmbo}

We won't be proving this. It's another theoretical result needing uniform convergence, like the fact that power series are analytic.

Much more interesting is that analytic functions admit Laurent series, and that the coefficients of the Laurent series are integrals! 

\begin{thmbo}{}{} Suppose $f(z)$ is analytic on $D = \{z\in\C|R_1 < |z-z_0| < R_2\}$. Let $r\in (R_1,R_2)$. Then on $D$, $f(z)$ is given by the Laurent series:

$$f(z) = \sum_{n = -\infty}^\infty a_n(z-z_0)^n$$

\noin where $a_n = \frac{1}{2\pi i}\int_{|z-z_0|=r} \frac{f(z)}{(z-z_0)^{n+1}}dz$.
\end{thmbo}

Proof will follow in the next lecture.