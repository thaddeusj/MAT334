\section*{Chapter 1 Problems}
\addcontentsline{toc}{section}{Chapter 1 Problems}

\begin{enumerate}
\item For each pair of complex numbers $z,w$ given below

\begin{enumerate}[i)]
\item $z = 3$, $w = 1 + i$
\item $z =2 - i $, $w =4 + 2i $
\item $z =(2 + 2i)(1 - 4i) $, $w = 5 - 6i$
\item $z = \frac{1 + i}{2 - i} $, $w = 2 - i $
\end{enumerate}

\noindent verify the following properties of complex algebra:

\begin{enumerate}[a)]
\item $z+w = w+z$ and $zw = wz$
\item $\RE(z + w) = \RE(z) + \RE(w)$ and $\IM(z + w( = \IM(z) + \IM(w)$
\item $\overline{zw} = \overline{z}\;\overline{w}$
\item $|zw| = |z||w|$
\item $|z| = |\overline{z}|$
\item $z\overline{z} = |z|^2$
\item $|\RE(z)| \le |z|$ and $|\IM(z)|\le |z|$

\end{enumerate}

\item For each of the following $z$, find the polar form for $z$, $z^3$ and $z^{17}$. Do not approximate.

\begin{enumerate}[a)]
\item $z =-1 - i $
\item $z = -1 + \sqrt{3}i $
\item $z = 3 + 4i$
\item $z = \frac{12 - 5i}{4}$
\item $z = 6 - i$

\end{enumerate}

\item The goal of this exercise is to verify a very important claim that we use all the time: that for any $z \ne 0$, any two arguments for $z$ differ by a multiple of $2\pi$.

\begin{enumerate}[a)] \item Suppose $\cos \theta = \cos \Psi$ and $\sin \theta = \sin \Psi$. Prove that $\cos(\theta - \Psi) = 1$.

\item For which $\theta \in \R$ is $\cos(\theta) = 1$? (This does not require proof.)

\item Conclude that if  $\cos \theta = \cos \Psi$ and $\sin \theta = \sin \Psi$, then $\Psi = \theta + 2k\pi$ for some integer $k$.

\item Conclude that if $z = r(\cos(\theta) + i\sin(\theta) = s(\cos(\Psi) + i \sin(\Psi)$ is written in polar form in two different ways, then $\Psi = \theta + 2k\pi$ and $r = s$. (Hint: remember that $r = |z|$ in the definition of polar form. What should $s$ be?)

\end{enumerate}

\item Prove the following:

\begin{enumerate}[a)]
\item For any $z \in \C$, $\overline{(\overline{z})} = z$.
\item For any $z,w \in \C$, $z\overline{w} - \overline{z}w$ is purely imaginary. (A complex number $z$ is called \textit{purely imaginary} if $z = 0 + yi$.)
\end{enumerate}

\item For each of the following complex numbers, find their principal argument:

\begin{enumerate}[a)]
\item $z = -1 + i$
\item $z = 2 - i$
\item $z = -11 - 15i$
\item $z = -11 + 15i$
\item $z = \frac{1}{3} + \frac{1}{\sqrt{3}}i$
\end{enumerate}


\item Solve the following equations:

\begin{enumerate}[a)]
\item $z^2 = 1 - i$
\item $z^3 = 8$
\item $z^4 = 12 - 5i$
\item $z^5 = z$
\item $z^6 = 27iz^3$
\end{enumerate}


\item Solve $z^n = w$ for each $w$ below, and each $n \in \N$.

\begin{enumerate}[a)]
\item $w = -1$
\item $w = 3 - i\pi$
\item $w = 2e^{7i}$
\item $w = 2e^{7}$
\end{enumerate}

\item Find the ranges of the following functions:

\begin{enumerate}[a)]
\item $f(z) = z$
\item $f(z) = z^3$
\item $f(x + iy) = xy$
\item $f(x + iy) =x^2 + i y^2$
\item $f(z) = 3z^2 + iz - 2$
\end{enumerate}

\item Find the range of $z^n$ for all $n \in \N$. (Try to turn this into a question about solving an equation.)

\item Solve the equations:

\begin{enumerate}[a)]
\item $e^z = 4$
\item $e^z = 2+2i$
\item $e^{(1+4i)z} = 13$
\item $e^{2z} + 4e^z = -3$
\end{enumerate}

\item Find the range of $f(z) = e^z$.

\item Let $w \in \C$. Show that there exists a branch $f$ of the 4th root function $z^\frac{1}{4}$ such that $f(w^4) = w$.

\item Suppose $w \in \C$ and $w \ne 0$. Futhermore, suppose $a^n = w$ for some $a \in \C$. Let $\omega_1,\dots, \omega_n$ be the $n$th roots of unity. Prove that the roots of $z^n = w$ are exactly $\{a\omega_j| j = 0, 1,2,\dots,n-1\}$.


\item The Quadratic Formula: Let $a,b,c \in \C$ and $a\ne 0$. Suppose $s_1$ and $s_2$ are the square roots of $b^2 - 4ac$. Prove that the roots of $az^2 + bz + c = 0$ are exactly $z_1 = \frac{-b + s_1}{2a}$ and $z_2 = \frac{-b + s_2}{2a}$.

\item Find and prove a formula relating $e^z$ and $e^{\OL{z}}$.


\item Let $z = re^{i\theta}$, where $\theta$ is chosen to be in $(-2\pi,0]$. Then define $f(z) = \sqrt{r}e^{i\frac{\theta}{2}}$.

\begin{enumerate}[a)]
\item Prove that $f(z)$ is a branch of the square root function.
\item Prove or disprove: $\OL{f(z)} = f(\OL{z})$.
\item Prove that $f(f(z))$ is a branch of $z^{\frac{1}{4}}$.
\end{enumerate}


\item We can also combine multi-valued functions using algebraic operations. We define addition and multiplication as follows: let $f, g$ be mutli-valued functions with domain $U$ and let $z\in U$. Then:
\begin{align*}
	f(z) + g(z) 	&= \{a+b: a\in f(z), b\in g(z)\}\\
	f(z)g(z) 	&= \{ab: a\in f(z), b\in g(z)\}
\end{align*}

Determine whether each of the following is a function or a multi-valued function:

\begin{enumerate}[a)]
\item $f(x + iy) = x^3 - 2xy$
\item $f(x + iy) = x^{\frac{1}{2}}y$
\item $f(z) = (z^2 - 3z)^\frac{1}{2}$
\item $f(z) = \left((z^2 - 3z)^\frac{1}{2}\right)^2$
\item $f(z) = [(z-1)^{\frac{1}{2}} + (z+1)^{\frac{1}{2}}][(z-1)^{\frac{1}{2}} - (z+1)^{\frac{1}{2}}]$
\end{enumerate}


\item Let $z_1 = 2 + i$ and $z_2 = -3 + i$. Assume that $z^a$ is the principal branch. Show that:

\begin{enumerate}[a)]
\item $\Log(z_1z_2) \ne \Log(z_1) + \Log(z_2)$
\item $(z_1z_2)^{1/2} \ne z_1^{1/2}z_2^{1/2}$
\item Show that $\left((z_1z_2)^{1/2}\right)^2 = z_1z_2$.
\item Is it true that $\left(z_1^{1/2}z_2^{1/2}\right)^2 = z_1z_2$?
\item Find $z_1^{z_2}$ and ${z_2}^{z_1}$.
\end{enumerate}


\item Let $a \in (0,\infty)$. Find all $z \in \C$ so that $\Arg(z^a) = a\Arg(z)$, where $z^a$ is the principal branch.

\item Let $a,b \in \C$. Is it true that $z^az^b = z^{a+b}$ and $\frac{z^a}{z^b} = z^{a-b}$, as multi-valued functions?

To be clear: if $f(z)$ and $g(z)$ are multi-valued functions, then:

$$f(z)g(z) = \{ab|a\in f(z), b\in g(z)\}$$

\item Show that $\Log(zw) = \Log(z) + \Log(w)$ if and only if $\Arg(z) + \Arg(w) \in (-\pi,\pi)$. Use this to show that $(zw)^a = z^aw^a$ if and only if $\Arg(z) + \Arg(w) \in (-\pi,\pi)$ or $a\in \Z$, where $z^a$ is the principal branch.

\item True or false: $e^{\log(z)} = z$. If true, give a brief argument. If false, give a counterexample.

\item True or false: $\log(e^z) = z$. If true, give a brief argument. If false, give a counterexample.

\item Find the range of $\sin(z)$ and of $\tan(z)$.



\end{enumerate}