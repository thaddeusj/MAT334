\chapter{Technical Proofs}

\section{The Leibniz Integral Rule}

Our proof of the Generalized Cauchy Integral Formula used a clever trick: differentiating in terms of $z_0$ and then swapping the derivative and the integral. To do this, we needed the Leibniz Integral Rule. We prove this result:

\begin{lem} Let $\gamma:[a,b]\rightarrow\C$ be a piecewise smooth curve. Suppose $f(w,z)$ is continuous in both $z$ and $w$ on some open set $R$ such that if $(w_0,z) \in R$ for some $z\in\C$ then $(w_0,\gamma(t))\in\R$ for all $t$. Further, suppose that $f_w(w,z)$ is continuous in both $z$ and $w$. Then:

$$\frac{d}{dw} \int_{\gamma} f(w,z)dz = \int_{\gamma} \frac{\partial}{\partial w} f(w,z)dz$$
\end{lem}

\begin{proof}
Define $F(w) = \int_\gamma f(w,z)dz$. We will show that $F$ is differentiable and that $F'(w_0) = \int_\gamma f_w(w_0,z)dz$.

Let $(w_0,z)\in R$. Since $R$ is open, there exists $r>0$ such that $B_r(w_0)\times\{z\}\subset R$. By our assumptions on $R$, this means that for any $h\in B_r(0)$, $(w_0+h,\gamma(t))\in R$ for all $t$. As such, if we assume $0<|h|<r$, we may consider $F(w_0+h) - F(w_0)$.

$$F(w_0+h) - F(w_0) = \int_\gamma f(w_0+h,z) - f(w_0,z)dz$$

Now, define $\ell_h(s) = w_0 + sh$ for $s\in [0,1]$. Then $\ell_h(s) \in B_r(w_0)$ for all $s$, so $(\ell_h(s),\gamma(t))\in R$. As such, we may consider $\int_{\ell_h} f_w(w,z)dw$. This integral is defined, and further our $\C$FTC gives:

$$\int_{\ell_h} f_w(w,z)dw = f(\ell_h(1),z) - f(\ell_h(0),z) = f(w_0+h,z) - f(w_0,z)$$

As such, we have that:

$$F(w_0+h) - F(w_0) = \int_\gamma\int_{\ell_h} f_w(w,z)dwdz$$

Using the definition of the integral, we get:

$$F(w_0+h) - F(w_0) = \int_a^b\int_0^1 hf_w(\ell_h(s),\gamma(t))\gamma'(t)dsdt$$

Note that $\RE\left(hf_w(\ell_h(s),\gamma(t))\gamma'(t)\right)$ and $\IM\left(hf_w(\ell_h(s),\gamma(t))\gamma'(t)\right)$ are continuous. (At this point, we should note that we really should assume $\gamma'(t)$ is continuous. Actually, in our use to prove GCIF, we can assume $\gamma'(t)$ is continuous by integrating over a circle and then deforming to the whole curve.)

As such, we can apply Fubini's theorem on the real and imaginary parts of this integral to swap the order of integration, giving:

$$F(w_0+h) - F(w_0) = \int_0^1\int_a^b hf_w(\ell_h(s),\gamma(t))\gamma'(t)dtds = \int_{\ell_h}\int_\gamma f_w(w,z)dzdw$$

We now consider the following limit:

$$\lim_{h\rightarrow 0} \frac{F(w_0+h) - F(w_0)}{h} - \int_{\gamma} f_w(w_0,z)dz$$

We wish to show this limit is $0$. We will do so from the definition. Let $\varepsilon > 0$. Since $w_0$ is constant, we have that $\int_{\ell_h}\int_{\gamma} f_w(w_0,z)dzdw = h\int_{\gamma} f_w(w_0,z)dz$. As such, we can rewrite this limit as:

\begin{align*}\frac{F(w_0+h) - F(w_0)}{h} - \int_{\gamma} f_w(w_0,z)dz&= \lim_{h\rightarrow 0} \frac{F(w_0+h) - F(w_0) - h\int_\gamma f_w(w_0,z)dz}{h}\\
&=  \frac{F(w_0+h) - F(w_0) - \int_{\ell_h}\int_\gamma f_w(w_0,z)dzdw}{h}\\
&=  \frac{\int_{\ell_h}\int_\gamma f_w(w,z)dz - \int_{\ell_h}\int_\gamma f_w(w_0,z)dzdw}{h}\\
&=  \frac{\int_{\ell_h}\int_\gamma f_w(w,z) - f_w(w_0,z)dzdw}{h}
\end{align*}

We now estimate using ML-estimation. Since $f_w$ is continuous as a function of $w$, there exists $\delta > 0$ such that if $|w-w_0| < \delta$, then $|f_w(w,z) - f_w(w_0,z)| < \frac{\varepsilon}{\mathrm{Length}(\gamma)+1}$. Note also that the length of $\ell_h$ is $|h|$. Then we estimate:

$$\left| \frac{\int_{\ell_h}\int_\gamma f_w(w,z) - f_w(w_0,z)dzdw}{h} \right| \le \frac{1}{|h|}|h| \max\left\{\left|\int_\gamma f_w(\ell_h(s),z) - f_w(w_0,z)dz\right|: s\in[0,1]\right\}$$

However, using ML estimate on this inner integral gives:

$$\left|\int_\gamma f_w(\ell_h(s),z) - f_w(w_0,z)dz\right| \le \mathrm{Length(\gamma)}\frac{\varepsilon}{\mathrm{Length}(\gamma)+1	}$$

So, all together, we get that:

$$\left| \frac{\int_{\ell_h}\int_\gamma f_w(w,z) - f_w(w_0,z)dzdw}{h} \right| < \varepsilon$$

\noindent whenever $0<|h| < \delta$. As such, $\lim_{h\rightarrow 0} \frac{F(w_0+h) - F(w_0)}{h} - \int_{\gamma} f_w(w_0,z)dz = 0$ by definition. So $F'(w_0) = \int_\gamma f_w(w_0,z)dz$, as desired.



\end{proof}




\section{The Extreme Value Theorem}\label{subsec:FTA}

To prove the Fundamental Theorem of Algebra, we used some technical results. In this subsection, we provide a proof. To begin, we need a couple of definitions.

\begin{defbo}{Closed Set}{closed}\index{Topology!closed set}
A set $K\subset \C$ is closed if $K^{c} = \{z\in \C|z\not\in K\}$ is open.\end{defbo}

\begin{ex}{Closed Balls}{} The closed ball $K = \{z\in \C| |z-z_0| \le R\}$ is closed.

Consider $w\not\in K$. Then $|w-z_0| > R$. Let $\varepsilon > 0$ such that $R + \varepsilon < |w- z_0|$. We claim that $B_{\varepsilon}(w) \subset K^c$. To see this, let $z\in B_{\varepsilon}(w)$. Then:

$$|z-z_0| = |(w-z) - (w-z_0)| \ge ||w-z| -|w-z_0|| = |w-z_0| - \varepsilon > R$$

So $z\not\in K$. As such, $B_{\varepsilon}(w) \subset K^c$.

Therefore, we have shown that for any $w\in K^c$, there is a ball $B_r(w)$ which is contained in $K^c$. So $K^c$ is open, and therefore $K$ is closed.
\end{ex}


\begin{defbo}{Cauchy Sequence}{} A sequence $z_1,\dots,z_n,\dots$ is called Cauchy if for all $\varepsilon > 0$, there exists some $N \in \N$ such that if $n,m > N$, then $|z_n - z_m| < \varepsilon$.
\end{defbo}

It is a neat fact, derived from the fact that $\C$ is topologically the same as $\R^2$, that $\C$ is complete: every Cauchy sequence in $\C$ converges.


\begin{thmbo}{Extreme Value Theorem}{EVT}\index{Extreme Value Theorem}
Suppose $f(z)$ is continuous on $\C$. Let $K\subset \C$ be closed and bounded. Then there exists $M\in \R$ such that $|f(z)| \le M$ for all $z\in K$. I.e., $f(z)$ is bounded on $K$.

\end{thmbo}

\begin{proof}
Suppose $f(z)$ is not bounded on $K$. I.e., for any $N\in \C$, there exists $z\in K$ such that $|f(z)| \ge N$.

Define $K_{n} = \{z\in \C| |f(z)| > 2^n\}$. We know that $K_{n}$ is non-empty for each $n$. For each $n$, choose $z_n\in K_n$.

Since $K$ is bounded, there exists some $a < b$ and $c < d$ so that $K\subset D = \{z\in \C| a \le \RE(z) \le b, c \le \IM(z) \le d\}$.

Now, there are an infinite number of $z_n$. (Since for any $z\in \C$ there exists $m$ with $|f(z)| < 2^m$, so for any $n$ there exists $m$ such that $z_n \ne z_j$ for all $j > m$.) 

We cut $D$ into four regions: 
$$\{z\in \C| a \le \RE(z) \le \frac{a+b}{2}, c \le \IM(z) \le \frac{c+d}{2}\}$$
$$\{z\in \C| a \le \RE(z) \le \frac{a+b}{2}, \frac{c+d}{2} \le \IM(z) \le d\}$$
$$\{z\in \C| \frac{a+b}{2} \le \RE(z) \le b, c \le \IM(z) \le \frac{c+d}{2}\}$$
$$\{z\in \C| \frac{a+b}{2} \le \RE(z) \le b, \frac{c+d}{2} \le \IM(z) \le d\}$$

Since there are an infinite number of $z_n$ and only four regions, one of these four rectangles contains an infinite number of the $z_n$. Set $D_1$ to be that region.

Repeat this procedure with $D_1$ to find another set $D_2 \subset D_1$ containing an infinite number of the $z_n$. Repeat to find $D_j$ such that: $D_j \subset D_{j-1}$ and $D_j$ contains an infinite number of these $z_n$.

From each $D_j$, choose one of the $z_n$ in that $D_j \cap K_j$, which we will call $w_j$. This intersection is non-empty since $D_j$ contains an infinite number of the $z_n$, and only finitely many of that $z_n$ are not in $K_j$.

I claim that the $w_j$ converge. To prove this, we will need to use a fact about $\C$: $\C$ is a complete metric space. This means that every Cauchy sequence in $\C$ converges. So, if we can prove that the sequence $w_1,w_2,\dots$ is Cauchy, then we know it converges.

This turns out to be easy. Let $\varepsilon > 0$. In a rectangle with side lengths $r,s$, the greatest distance between any two points in the rectangle is given by considering the distance between two opposite vertices. This is $\sqrt{r^2 + s^2}$. By our construction, the rectangle $D_j$ has side lengths $2^{-j}(b-a)$ and $2^{-j}(d-c)$. Therefore, choose $N$ such that $2^{-N}\sqrt{(b-a)^2 + (d-c)^2} < \varepsilon$.

Then for any $n,m > N$, we have that $w_n \in D_n \subset D_N$ and $w_m\in D_m \subset D_N$. As such:

$$|w_n - w_m| \le  2^{-N}\sqrt{(b-a)^2 + (d-c)^2} < \varepsilon$$

So the sequence is Cauchy, and hence converges to some $w$. Furthermore, since $K$ is closed, we know that $w\in K$.

To finish up, we use that $f(z)$ is continuous. This tells us that:

$$|f(w)| = |f(\lim_{j\rightarrow \infty} w_j)| = |\lim_{j\rightarrow \infty} f(w_j)| = \lim_{j\rightarrow \infty} |f(w_j)|$$

However, we know that for any $N \in \N$, if $j > N$ then $w_j\in D_N \cap K_N$, and so $|f(w_j)| > 2^N$. As such, $|f(w)| = \lim_{j\rightarrow \infty} |f(w_j)| = \infty$. However, this is not possible since $f$ is continuous at $w$. Contradiction.

Therefore, $f$ is bounded on $K$.

\end{proof}


\section{Series}

\begin{thmbo}{}{absConv}Absolutely convergent series converge.\end{thmbo}

\begin{proof} Suppose $\sum_{n = k}^\infty a_n$ converges absolutely. To show that it converges, we need to show that $(S_n)_{n=k}^\infty$ converges. As mentioned in subsection \ref{subsec:FTA} of these appendices, this is equivalent to showing that the sequence is Cauchy.

I.e., we need to show that we can force $|S_n - S_m| = |\sum_{j = m+1}^n a_j|$ to be very small.

Let $\varepsilon > 0$. Since $\sum_{n = k}^\infty |a_n|$ converges, there exists $N\in \N$ such that $n > m > N$ implies that $|T_n - T_m| < \varepsilon$ where $T_n = \sum_{j = k}^n |a_n|$.

Now, $|T_n - T_m| = |\sum_{j = m+1}^n |a_j|| = \sum_{j = m+1}^n |a_j|$.

By the triangle inequality:

$$|\sum_{j = m+1}^n a_n| \le \sum_{j = m+1}^n |a_j|$$

And so $|S_n - S_m| \le |T_n - T_m|$. Therefore, if $n,m > N$, then $|S_n - S_m| \le |T_n - T_m| < \varepsilon$. This proves that the sequence $(S_n)_{n=k}^\infty$ is Cauchy, and therefore converges as well.
\end{proof}

For some of our theorem on analytic functions, we need to discuss the notion of uniformly convergent series.

\begin{defbo}{Uniform Convergence}{} Suppose $(f_n)_{n = k}^\infty$ is a sequence of functions, defined on some set $S$. We say that $f_n$ converges uniformly to $f$ on $S$ if:

$$\forall \varepsilon > 0, \exists N \in \N \text{ such that } n > N \implies |f_n(z) - f(z)| < \varepsilon \forall z\in S$$

A uniformly convergent series is a series whose partial sums converge uniformly.
\end{defbo}

Notice the order of the quantifiers here. This is saying that $N$ doesn't depend on $z$. So not only does the series converge to $f$ at each point, but the series converges at roughly the same speed. More precisely, there's some lower bound to how fast the series converges at each point.

Why is this important? Well, it let's me get away with swapping some integrals and limits. In particular, it lets us do the following:

\begin{thmbo}{}{uniform}Suppose $f_n$ are all continuous on the piecewise smooth curve $\gamma$ and $f_n \rightarrow f$ uniformly. Then $\lim_{n\rightarrow \infty} \int_{\gamma} f_n(z)dz = \int_\gamma f(z)dz$.
\end{thmbo}

\begin{proof} Since $\left|\int_{\gamma} f_n(z)dz - \int_{\gamma} f(z)dz\right| \le \int_{\gamma} |f_n(z) - f(z)|dz$, we prove that:

$$\lim_{n\rightarrow \infty} \int_{\gamma} |f_n(z) - f(z)|dz = 0$$

Let $\varepsilon > 0$. Then there exists $N \in \N$ such that $n\ge N$ implies that $|f_n(z) - f(z)| \le \frac{\varepsilon}{L(\gamma)}$, where $L(\gamma)$ is the length of $\gamma$.

Now, by M-L estimation, $\int_{\gamma} |f_n(z) - f(z)|dz \le \frac{\varepsilon}{L(\gamma)}L(\gamma) = \varepsilon$.

So, we have shown that $\forall\varepsilon > 0$ there exists some $N\in \N$ such that $n > N$ implies $\int_{\gamma} |f_n(z)- f(z)|dz = \left| \int_{\gamma} |f_n(z) - f(z)| dz - 0 \right| < \varepsilon$. As such, $\lim_{n \rightarrow \infty} \int_{\gamma} |f_n(z) - f(z)|dz = 0$.

By the squeeze theorem, $\lim_{n \rightarrow \infty} \left|\int_{\gamma} f_n(z)dz - \int_{\gamma} f(z)dz\right| = 0$, and so $\lim_{n\rightarrow \infty} \int_{\gamma} f_n(z)dz = \int_{\gamma} f(z)dz$.
\end{proof}

All that remains for our purposes is to show that $\sum_{n = 0}^\infty z^n$ converges uniformly on $|z| \le R$ for any $R < 1$.

\begin{thmbo}{}{1/1-z} Let $0 \le R < 1$. Then $\sum_{n = 0}^\infty z^n$ converges uniformly to $\frac{1}{1-z}$ on $|z| \le R$.\end{thmbo}

\begin{proof} Let $D = \{z\in\C||z| \le R\}$. Let $S_n(z) = \sum_{j = 0}^n z^j$. Recall that $S_n(z) = \frac{1 - z^{n+1}}{1-z}$. So, to show that the sum converges uniformly, we need to show that $\frac{1- z^{n+1}}{1-z}$ converges uniformly to $\frac{1}{1-z}$.

Notice that $|S_n(z) - \frac{1}{1-z}| = \frac{z^{n+1}}{1-z}$. Now, since $|z| \le R < 1$, we have that $|1 - z| \ge 1 - R > 0$. As such, $\frac{z^{n+1}}{1-z} \le \frac{R^{n+1}}{1 - R}$.

Let $\varepsilon > 0$. Since $\lim_{n\rightarrow \infty} \frac{R^{n+1}}{1-R} = 0$, there exists $N\in \N$ such that if $n > N$, then $\frac{R^{n+1}}{1-R} < \varepsilon$.

However, this also gives that $|S_n(z) - \frac{1}{1-z}| \le \frac{R^{n+1}}{1-R} < \varepsilon$ for all $z\in D$. As such, the series converges uniformly to $\frac{1}{1-z}$ on $D$.
\end{proof}

In actuality, we need that $\sum_{n = 0}^\infty f(z) \frac{(w-z_0)^n}{(z-z_0)^{n+1}}$ where $|w-z_0| < |z-z_0|$ converges uniformly to $\frac{f(z)}{z-w}$ on a simple, closed curve $\gamma$. Since $f(z)$ is bounded on the curve, this is a simple modification of the above argument.