\chapter{Selected Answers and Hints}

\section{Chapter 1}

\begin{enumerate}

\item Note that for each of these properties, you can use a computer algebra system (such as Wolfram Alpha) to directly verify your computations. In general, I will not post answers to questions you can verify easily.

\item \begin{enumerate}[a)]
\item $\sqrt{2}e^{i\frac{5\pi}{4}}$, $\sqrt{2}^3e^{i\frac{15\pi}{4}}$, and $\sqrt{2}^{17}e^{i\frac{85\pi}{4}}$
\item $2e^{i\frac{2\pi}{3}}$, $2^3e^{i\frac{6\pi}{3}} = 8$, and $2^{17}e^{i\frac{34\pi}{3}}$
\item $5e^{i\arctan(4/3)}$, $125 e^{3i\arctan(4/3)}$, and $5^{17}e^{17i\arctan(4/3)}$
\item $\frac{13}{4}e^{i\arctan(-5/12)}$, $\left(\frac{13}{4}\right)^3e^{i\arctan(-5/12)}$, and $\left(\frac{13}{4}\right)^{17}e^{17i\arctan(-5/12)}$
\item $\sqrt{37}e^{i\arctan(-1/6)}$, $\sqrt{37}^3e^{3i\arctan(-1/6)}$, and $\sqrt{37}^{17}e^{17i\arctan(-1/6)}$

\end{enumerate}

\item

\begin{enumerate}[a)]
\item There is a trig identity which will help you here.
\item $\theta = 2k\pi$ for $k\in\Z$
\item Combine parts (a) and (b).
\item Start by showing $r = s$.
\end{enumerate}

\item Work straight from the definitions. Take $z = a+bi$ (and in part (b), $w = c+di$) and then compute the quantities being asked about.

\item \begin{enumerate}[a)]
\item $\frac{3\pi}{4}$
\item $\arctan(-1/2)$
\item $\arctan(15/11) - \pi$
\item $\arctan(-15/11) + \pi$
\item $\frac{\pi}{3}$

\end{enumerate}

\item \begin{enumerate}
\item $z = \pm \sqrt[4]{2}e^{-i\pi/8}$
\item $z = 2$, $z = -1\pm \sqrt{3}i$
\item $z = \sqrt[4]{13}e^{i(\arctan(-5/12)/4 + k\pi/2)}$ for $k = 0,1,2,3$
\item $z = 0, 1, -1, i, -i$
\item $z = 0, -3i, \pm\frac{3\sqrt{3}}{2} + \frac{3}{2}i$
\end{enumerate}

\item \begin{enumerate}[a)]
\item $z = e^{i\frac{(2k+1)\pi}{n}}$ for $k = 0,1,\dots, n-1$
\item $z = \sqrt{9 + \pi^2}^{\frac{1}{n}}e^{i(\frac{\arctan(-\pi/3) + 2k\pi}{n})}$ for $k = 0,1,\dots, n-1$
\item $z = 2^{\frac{1}{n}}e^{i\frac{7 + 2k\pi}{n}}$ for $k = 0,1,\dots,n-1$
\item $z = (2e^7)^{\frac{1}{n}}e^{i\frac{2k\pi}{n}}$ for $k=0,1,\dots,n-1$

\end{enumerate}

\item \begin{enumerate}\item $\C$
\item $\C$
\item $\R$
\item $\{a+bi: a,b\ge 0\}$
\item $\C$
\end{enumerate}

\item $\C$

\item \begin{enumerate}
\item $z = \ln 4 + i2k\pi$ for $k\in \Z$
\item $z = \ln(\sqrt{8}) + i(\frac{\pi}{4} + 2k\pi)$ for $k\in\Z$
\item $z = \frac{\ln(13) + i2k\pi}{1+4i} = \frac{\ln(13) - 8k + (4\ln(13) + 2k)i}{17}$ for $k\in\Z$
\item $z = i(\pi + 2k\pi)$ and $z = \ln(3) +i(\pi + 2k\pi)$ for $k\in \Z$. (Hint: factor.)

\end{enumerate}

\item $\C \setminus\{0\}$

\item You will need to choose your angles so that an argument for $w$ is in the resulting range.

\item First, check that the roots we've given are actually roots. Second, you need to check that they are the only roots. To show this, consider what the maximum number of roots a polynomial of degree $n$ can have.

\item You could prove this by directly verifying that $z_1$ and $z_2$ are roots. You could also prove this by completing the square.

\item $e^{\overline{z}} = \overline{e^z}$

\item \begin{enumerate}
\item Check the defining property of a branch of the square root.
\item False.
\item Check the defining property of a branch of the fourth root, and verify that $f(f(z))$ is a function.

\end{enumerate}

\item Exactly two of these are functions.

\item \begin{enumerate}
\item Compare the imaginary parts of each side.
\item This follows from your computations in part (a).
\item Use the definition of $z^{\frac{1}{2}}$.
\item Use the definition of $z^{\frac{1}{2}}$ and the commutativity of multiplication.
\item $$z_1^{z_2} = e^{-3\ln\sqrt{5} - \arctan\left(\frac{1}{2}\right)}e^{i(-3\arctan\left(\frac{1}{2}\right) + \ln\sqrt{5})}$$ 

$$z_2^{z_1} = e^{2\ln\sqrt{10} - (\arctan\left(\frac{-1}{3}\right) + \pi)}e^{i(2\arctan\left(\frac{-1}{3}\right) + 2\pi + \ln\sqrt{10})}$$

\end{enumerate}

\item $\theta \in \left(\frac{-\pi}{a}, \frac{\pi}{a}\right)$

\item No, these are not the same functions. One example you can look at is $i^{i}i^{2i}$.

\item For the power question, showing that $(zw)^a = z^aw^a$ implies $\Arg(z) + \Arg(w) \in (-\pi,\pi)$ or $a\in \Z$ is tricky. 

It will be useful to note that either $\Arg(z) + \Arg(w) = \Arg(zw)$ or $\Arg(z) + \Arg(w) = \Arg(zw) \pm 2\pi$. It will also be useful to recall that if $e^{i\theta} = e^{i\phi}$, then $\theta -\phi = 2k\pi$ for some $k\in\Z$.

\item True.

\item False.

\item The range of $\sin(z)$ is $\C$. The range of $\tan(z)$ is $\C\setminus\{\pm i\}$.

\end{enumerate}



\section{Chapter 2}

\begin{enumerate}

\item \begin{enumerate}[a)]
	\item $-1$
	\item Does not exist
	\item Does not exist
	\item Does not exist
	\item $1$
	\item Does not exist
\end{enumerate}

\item \begin{enumerate}[a)]
	\item $\{iy: y\in [0,\infty)\}$
	\item $(-\infty,-1] \cup \{re^{i\frac{\pi}{3}}: r\ge 1\} \cup \{re^{i\frac{5\pi}{3}}: r\ge 1\}$
	\item $(-\infty,1]$
	\item $(-\infty,1]$
	\item $(-\infty,0] \cup [1,\infty)$
	\item No branch cut. This function is continuous on $\C$.
\end{enumerate}

\item For each of these, verify that $u_x = v_y$ and $u_y = -v_x$ on their domains, and that their domains are open.

\item Use either $f'(z) = u_x + iv_x$ or $f'(z) = v_y - iu_y$. You should have these already from the previous part. The simplify to get the expected answers:

\begin{enumerate}[a)]
	\item $\frac{-1}{z^2}$
	\item $e^{z+1}$
	\item $2ze^{z^2}$
	\item $\frac{z^2-2z}{(z-1)^2}$
	\item $\cos(z)$
	\item $-\sin(z)$
	\item $\cosh(z)$
\end{enumerate}

\item $f'(z) = \Log(2)f(z)$
\item $f'(z) = \Log(i)f(z)$
\item If the branch of $a^z$ is given by the branch $\log_0$ of the logarithm, then $f'(z) = \log_0(a)f(z)$.
\item $\Arccos(z) = -i\Log(z+(z^2-1)^\frac{1}{2})$ and $\Arcsin(z) = -i\Log(iz + (1-z^2)^\frac{1}{2})$.
\item Determine when $z^2 - 1\in (-\infty,0]$.
\item 
\begin{enumerate}[a)]
	\item $z + (z^2-1)^\frac{1}{2} \not\in (-\infty,0]$.
	\item Move $z$ to the right hand side and then square both sides.
	\item You can use first year calculus to maximize this.
	\item Notice here that $(z^2-1)^\frac{1}{2} = \sqrt{z^2-1}$ since $z^2 - 1 \ge 0$. Show that $z + \sqrt{z^2-1} \le -1$.
\end{enumerate}

\item
\begin{enumerate}[a)]
	\item $-i\ln(2+\sqrt{3})$
	\item $-i\ln(\sqrt{2}-1)$
\end{enumerate}

\item 
\begin{enumerate}[a)]
	\item No.
	\item Yes.
	\item No.
	\item Yes.
	\item Yes.
	\item Yes.
	\item Yes.
	\item Yes.
\end{enumerate}

\item \begin{enumerate}[a)]
\setcounter{enumii}{1}
	\item $v(x,y) = x^2 + 8xy - y^2 + C$
\setcounter{enumii}{3}
	\item $v(x,y) = 3x^2y - y^3 + C$
	\item $v(x,y) = 4x^3y - 4xy^3+ C$
	\item $v(x,y) = \frac{-2y}{x^2+y^2} + C$
	\item $v(x,y) = -\ln(x^2+y^2) + C$
	\item $v(x,y) = 2\arctan\left(\frac{y}{x}\right) + C$
\end{enumerate}

\item This is false. Consider the example $u = x$ and $v = y$.

\end{enumerate}

\section{Chapter 3}

\begin{enumerate}

\item\begin{enumerate}[a)]
\item Closed. Orientation not specified.
\item Not closed.
\item Not closed.
\item Closed, orientation not specified. We have not said if it is the upper or lower semicircle.
\item Closed, positively oriented.
\item Closed, positively oriented.
\item Closed, but not simple. Not possible to determine orientation.
\end{enumerate}

\item Parametrize the curves first. Use question 4 to parametrize each part separately and compute their integrals separately.

\item Do a $u$ substitution.

\item Do a $u$ substitution. Keep in mind that you have $\gamma_1: [a,b]\rightarrow \C$ and $\gamma_2:[c,d]\rightarrow \C$, but you don't know $b = c$, so you can't just integrate from $a$ to $d$.

\item Use the previous 2 questions.

\item \begin{enumerate}[a)]
\item Yes.
\item No.
\item Yes.
\item No. $z^2 - z$ is negative on $[0,1]$, which has non-empty intersection with the inside of $\gamma$.
\item No. $z^2 - z$ is positive on $[-1,0]$, which has non-empty intersection with the inside of $\gamma$.
\item No. Be careful: $e^{\Log(z)}$ is not just $z$. It is only $z$ wherever it is defined, which does not include $(-\infty,0]$.
\end{enumerate}

\item \begin{enumerate}[a)]
\item $\frac{8\pi^3}{3}$
\item $0$
\item $\Log(-3i) - \Log(2)$
\item Let $\log_0$ be given by the argument $\arg_0(z) \in (\theta,\theta + 2\pi)$. Then $\int_{\gamma} \frac{1}{z}dz = \log_0(b) - \log_0(a)$.
\item $\frac{1}{2} - \frac{1}{3i}$
\item $-3i\Log(-3i) + 3i -2\Log(2) -2$
\end{enumerate}

\item \begin{enumerate}[a)]
\item Yes.
\item No.
\item No.
\item No.
\end{enumerate}

\item \begin{enumerate}[a)]
\item No. Curve isn't closed.
\item No. Curve contains the singularity for $\cos\left(\frac{1}{z}\right)$.
\item Yes. The singularity is outside the curve.
\item No. $e^{2z}\Log(z)$ is not defined on the whole curve.
\item Yes. $e^{2z}\Log(z)$ is analytic on $\C\setminus(-\infty,0]$.
\item Yes. $e^2$ is entire.
\end{enumerate}

\item \begin{enumerate}[a)]
\item Yes.
\item Yes.
\item No.
\item Yes.
\item Yes, but not helpful.
\item No.
\item Yes.
\end{enumerate}


\item Look at the curve $\gamma_1 - \gamma_2$.

\item \begin{enumerate}[a)]
\item None.
\item $\pm i$
\item $\pm i$
\item $\frac{3}{2} + \frac{3\sqrt{3}}{2}i$
\end{enumerate}

\item \begin{enumerate}[a)]
\setcounter{enumii}{1}
\item Neither are.
\item Neither are.
\item It is.
\end{enumerate}

\item \begin{enumerate}[a)]
\item It has 3 inside this circle.
\item It has 5 inside this circle.
\item It has 3 inside this circle.
\item It has 3 inside this circle: $0, -\pi/2,$ and $\pi/2$.
\item It has 5 inside this circle.
\end{enumerate}

\item None of them are possible. They all have roots between the two curves.

\item \begin{enumerate}[a)]
\item $0$
\item $0$
\item $2\pi i \cos\left(\frac{1}{2}\right)e^{\frac{1}{2}}$
\item $2\pi i \ln(2)$
\item $-\frac{4\pi i}{3}$
\item $\frac{2^{2n}(2n-2)!\pi i}{((n-1)!)^23^{2n-1}}$
\item $0$
\end{enumerate}

\item \begin{enumerate}[a)]
\item $\int_{\gamma} f(z)dz = \sum_i \int_{\gamma_i} f(z)dz$.
\item Use part $a)$ with $\gamma_i =\gamma$.
\end{enumerate}

\item Induct using the previous question.

\item You will get twice your answer from problem 16.

\item \begin{enumerate}[a)]
\item $0$
\item $0$
\item $-\frac{\pi i}{2}$
\item $\frac{\pi i}{24}$
\item $-\frac{2\pi i \cos(1)}{\sin^2(1)}$
\item $0$
\item $0$
\item $\frac{\pi e \sin(1) i}{2}$

\end{enumerate}

\end{enumerate}


\section{Chapter 4}

\begin{enumerate}
	\item \begin{enumerate}
		\item Converges.
		\item Converges.
		\item Diverges
		\item Diverges
		\item Diverges
		\item Converges
		\item Diverges
		\item Diverges
		\item Converges
		
	\end{enumerate}
	
	\item \begin{enumerate}
		\item $R = 1$
		\item $R = 1$
		\item $R = 2$
		\item $R = |w|$
		\item $R = \infty$
		\item $R = \frac{1}{2}$		
	\end{enumerate}
	
	\item The condition is that $R_1 < R_2$.
	
	\item Use the Fundamental Theorem of Algebra.
	
	\item \begin{enumerate}
		\item Polynomials are already power series centered at $0$.
		\item $\ds \sum_{n = 0}^\infty \frac{2^nz^n}{n!}$
		\item $\ds \sum_{n = 0}^\infty \frac{e^ba^nz^n}{n!}$
		\item $\ds \sum_{n = 0}^\infty \frac{a^n\left(z-\frac{-b}{a}\right)^n}{n!}$
		\item $\ds \sum_{n = 0}^\infty \frac{(-1)^{n}z^{6n+4}}{(2n+1)!}$
		\item $\ds \sum_{n = 0}^\infty i^{n}z^{3n} + \sum_{n=0}^\infty i^{n}z^{3n+2}$
		\item $\ds \sum_{n = 0}^\infty (-1)^n \left(\frac{3}{(2n)!} + \frac{1}{(2n+1)!}\right)z^{6n+3}$
		\item $\ds \sum_{k = 0}^\infty nz^{(k+1)n - 1}$
		\item $\ds \sum_{n = 0}^\infty (-1)^n(z-1)^n$
		\item $\ds \sum_{k = n-1}^\infty (-1)^{k+n-1} \frac{k!}{(n-1)!(k-n+1)!}z^{k-(n-1)}$
		\item $\ds \sum_{n = 0}^\infty \frac{z^n}{a^{n+1}}$
		\item $\ds \sum_{n = 0}^\infty \frac{z^{2n}}{a^{2n+2}}$
		\item $\ds \sum_{n = 0}^\infty \frac{(-1)^n}{n+1}(z-1)^{n+1}$
		\item $\ds \sum_{n = 0}^\infty \frac{(-1)^n}{n+1}(z-1)^{n+1} + i4\pi$
	
	\end{enumerate}
	
	\item \begin{enumerate}
		\item $R = \infty$
		\item $R = \infty$
		\item $R = \infty$
		\item $R = \infty$
		\item $R = \infty$
		\item $R = 1$
		\item $R = \infty$
		\item $R = 1$
		\item $R = 1$
		\item $R = 1$
		\item $R = |a|$
		\item $R = |a|$
		\item $R = 1$
		\item $R = 1$
	\end{enumerate}
	
	\item \begin{enumerate}
		\item $e^{iz}$
		\item $-\Log(1-iz)$
		\item $\frac{iz}{1-(iz)^2}$
		\item $z^2\cos\left(\frac{z^2}{\sqrt{3}}\right)$
		\item $\frac{-1}{z}$
		\item $\frac{1}{1-2z}$
		\item $-\Log(1-z) - e^z - 1$
	\end{enumerate}

	\item \begin{enumerate}
		\item $\sin(\pi^3)$	
		\item $e^{\sqrt{\pi}}$
		\item $0$
		\item Diverges.
	
	\end{enumerate}

	\item One direction is immediate. In the other direction, write $f$ as a power series.
	
	\item Look at the difference of the two series.
	
	\item \begin{enumerate}
		\item $z = 0 , 1, -1$, all of order $1$.
		\item $z = 0$ of order $2$ and $z = k\pi$ for $k\in \Z$, $k\ne$ of order 1.
		\item $z = k\pi$ and $k\pi + \frac{\pi}{2}$ for $k\in \Z$, of order 1.
		\item $z = (2k+1)\pi$ for $k\in \Z$, of order 1.
		\item This function has no zeroes.
		\item $z = k\pi$ for $k\in \Z$, $k\ne 0$, of order 1.
		\item This function has no zeroes.
		\item All $z$ with $|z| > 1$. These are zeroes of infinite order.
		\item $z = 1$ of order 2.
		\item $z = 0$ of order $n$, $z = 2k\pi i$ for $k\in \Z$, $k\ne 0$, of order 1.
	\end{enumerate}
	
	\item \begin{enumerate}
		\item $z = 0$, a removable discontinuity
		\item $z = 0$, an essential singularity
		\item $z = -2$, a pole of order 2
		\item $z = -2 \pm i$, each simple poles
		\item $z = 2k\pi i$ for $k\in \Z$, each simple poles
		\item $z = k\pi$ for $k\in \Z$, each simple poles
		\item If $n \ge m$, a removable discontinuity at $z = 0$. If $n < m$, a pole of order $m-n$ at $z = 0$.
		\item No isolated singularities.
		\item No isolated singularities.
	
	\end{enumerate}

	\item \begin{enumerate}
		\item $0$
		\item $1$
		\item $0$
		\item $\Res(f;-2+i) = \frac{1}{2i}$ and $\Res(f;-2-i) = \frac{-1}{2i}$
		\item $1$
		\item $(-1)^k$
		\item For $m\le n$, the residue is $0$. For $m > n$, it becomes a mess. Expanding the power of $\sin(z)$ is not reasonable, and other methods involving taking the derivatives of the expression become unwieldy.
		
		
	
	\end{enumerate}
	
	
	\item \begin{enumerate}
		\item This follows right from the definition.
		\item $n < k$ of order $k-n$
		\item $z^k$
		\item $\sum_{k = 0}^\infty z^{a_k}$
	\end{enumerate}
	
	\item	\begin{enumerate}
		\item $\ds \sum_{n = 0}^\infty \frac{(-1)^n}{(2n+1)!z^{2n+1}}$
		\item $\ds \sum_{n = 0}^\infty \frac{(-1)^n}{2^{2n+2}(2n+1)!z^{2n+2}}$
		\item $1$
		\item $\ds \sum_{n = 0}^\infty \frac{(-1)^n}{2^{2n}(2n)!z^{2n}}$
		\item $\ds \sum_{n = 0}^\infty \frac{z^{2n-1}}{n!}$
		\item $\ds \sum_{k = 0}^\infty \frac{(-1)^n z^{(2k+1)n-m}}{(2k+1)!}$
		\item $\ds \sum_{n = 0}^\infty -z^{n-1}$
		\item $\ds \sum_{n = 0}^\infty z^{-n+2}$
		\item Already a Laurent series centered at $1$.
		\item $\ds \sum_{n = 1}^\infty nz^{n-1}$
		\item $\ds \sum_{n = 0}^\infty \frac{n}{z^{n+1}}$
		\item $\ds \sum_{n = 0}^\infty \frac{e(z-1)^{n-1}}{n!}$
		\item $\ds \sum_{n = 0}^\infty \frac{(-1)^nz^n}{n!}$
		\item $2i$
	\end{enumerate}
	
	
	\item \begin{enumerate}
		\item $2\pi i$
		\item $0$
		\item $\frac{2\pi i e}{3} + \frac{2\pi i e^{\frac{-1 + \sqrt{3}i}{2}}}{3e^{\frac{-1 - \sqrt{3}i}{2}}} + \frac{2\pi i e^{\frac{-1 - \sqrt{3}i}{2}}}{3e^{\frac{-1 + \sqrt{3}i}{2}}}$
		\item $2\pi i \left( \frac{1}{2} -\frac{1}{8} - \frac{2^i}{4i} + \frac{2^{-i}}{4i}\right)$ where $2^z$ is chosen to be some branch.
		\item $\frac{12}{5} \sum_{k = 0}^4 e^{\frac{2k \pi i}{5}}$
		\item $2\pi i$ if $n = 1$ and $0$ if $n  > 1$		
		\item $0$
		\item $4\pi i$
		\item $2\pi i$
		\item $0$
		\item $0$
	\end{enumerate}
	
	\item This follows immediately from the formula for the coefficients in a Laurant series.
	
	\item 
	\begin{enumerate}
		\item $\frac{2\pi}{3}$
		\item $\pi$
		\item $\pi$
		\item $\left(1- \frac{1}{\sqrt{2}}\right)\pi$
		\item This does not exist.
		\item $\frac{3\pi}{16}$
	
	\end{enumerate}
	
	
	\item 
	\begin{enumerate}
		\item $\frac{\pi}{2e}$
		\item $\frac{\pi(4\sin(1) + \cos(1))}{e}$
		\item $\frac{7\pi\cos(1)}{8e}$
		\item $\sqrt{2}e^{-\sqrt{2}}\pi - \frac{\pi}{e}$
		\item Don't worry about this one. It's a little bonus challenge.
	
	\end{enumerate}
\end{enumerate}

